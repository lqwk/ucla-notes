\documentclass{scrartcl}

\KOMAoptions{
  fontsize=8pt,
}

\addtokomafont{disposition}{\rmfamily}

\usepackage{bm}
\usepackage{graphicx}
\usepackage{mathtools}
\usepackage{amsmath}
\usepackage{float}
\usepackage{ragged2e}
\usepackage[font=footnotesize,labelfont=bf]{caption}
\usepackage{geometry}
\geometry{letterpaper, left=.1in, right=.1in, top=.05in, bottom=.08in}
\usepackage{multicol}

\DeclareOldFontCommand{\bf}{\normalfont\bfseries}{\mathbf}

\pagenumbering{gobble}
\setlength{\parskip}{0.7em}
\setlength\parindent{0pt}

\begin{document}
\raggedright
\begin{multicols*}{3}


{\bf 1. Introduction: What is Myth?}

{\bf muthos}: alternative spelling: mythos. The word `muthos' is used in Homer to refer to a formal, authorative speech, often before an audience. Gradually, however, the meaning of muthos changes over time. In the Classical period, Plato contrasted muthos with logos. In this, he paved the way for the modern interpretation of myth as an untrue story.

{\bf trado}: The Latin verb `trado, tradere' means ``to hand down". It is where our word tradition comes from.


{\bf 2. Hesiod: The Creation of the Greek Cosmos}

{\bf Archaic Period}: The Archaic Period of Greece runs from approximately the 8th to the 6th centuries, BCE, and predates the Classical Period. In the archaic period, Greek epic and lyric poetry flourished. Homer, Hesiod, and the author of the Homeric Hymns all composed in the archaic period. See further: Powell, A Short Introduction to Classical Myth, pp. 55-57.

{\bf Chaos}: The first principle, from which the cosmos emerges. See Theogony lines 115 following.

{\bf Epic}: The earliest surviving form of Greek literature, epic is a form of poetry that flourished in the archaic period. It began as an oral form (Homer recited the Iliad and the Odyssey from memory) but was later written down. See further Homer, Hesiod, Theogony.

{\bf Gaia}: Gaia is the Greek goddess of the Earth. She is born after Chaos (Theogony 115 ff.). She partners with Ouranos and gives birth to the race of Titans. Overpowered by the weight of Ouranos, Gaia gives her unborn son, Kronos, a sickle with which to castrate his father.

{\bf Muses}: Daughters of Memory, the Muses appear at the beginning of the Theogony, and at the beginning of the Iliad and the Odyssey. They are invoked by the epic poet to help him begin and remember his song.

{\bf Olympians}: The children of the Titans. The most important Olympian is Zeus (king of the gods). Aphrodite is also an Olympian, although she is technically born in an earlier generation. The Olympians get their name from Mt. Olympos, where they live as a somewhat happy, somewhat unhappy family. The Greeks liked to think of the Olympians as being 12 in number, but in practice the number was greater. They are Zeus, Hera, Demeter, Poseidon, Hestia, Hades, Hephaestus, Athena, Apollo, Artemis, Ares, Hermes, Persephone, Aphrodite, Dionysos (we first meet them in Hesiod's Theogony, but we will talk alot more about this group of gods as we move through the course).

{\bf Ouranos}: Son and mate of Gaia, Ouranos is the god of the sky. He is castrated by Kronos in the early part of the Theogony.

{\bf Primordial Gods}: These are the very first gods to populate the universe in Hesiod's Theogony. They include Chaos, Gaia, and Ouranos.

{\bf Theogony}: Epic poem about the origins of the cosmos (a `cosmogony' (birth of the cosmos) as well as a `theogony' (birth of the gods)); composed by Hesiod; Hesiod appears as a character in his own poem in the prologue.

{\bf Titans}: The children of Ouranos and Gaia as described in Hesiod's Theogony. Although they live forever, Hesiod refers to them as ``the gods from before" (that is, before the reign of Zeus - see Olympians).


{\bf 3. Battle, Succession, and Interpretation}


{\bf Athena}: (alt. sp = Athene). Goddess of wisdom. Emerges from Zeus' head in the Theogony. Special protector of Odysseus and Telemachus in the Odyssey (appears to Telemachus in the guise of Mentes and Mentor in bks 1-4).

{\bf Kronos}: The son of Ouranos, Kronos tries to fight off succession by swallowing his own children. He is a god of the older generation (a Titan), married to Rhea, and father of Zeus and many of the other Olympians. His story is told in the Theogony. In the Roman tradition, Kronos is renamed `Saturn'. See further Goya, ``Saturn Devouring His own Children (under images)."

{\bf Metis}: Greek word meaning ``cunning". The first wife of Zeus in the Theogony, whom Zeus swallows when she is pregnant with Athena.

{\bf Rhea}: Wife of Kronos in the Theogony. Mother of Zeus, whom she hides away to save from Kronos.

{\bf Typhoios}: Monstrous child of Gaia and Tartaros. Typhoeus challenges Zeus' rule in the Theogony and is eventually defeated by him. Compare lecture 7, where an alternative version of the birth of `Typhaon' (same god) is given in the Homeric hymn to Apollo.

{\bf Zeus}: Supremely powerful ruler of gods and men. Zeus is the son of Kronos and the leader of the Olympians. The Theogony tells the story of his succession and celebrates his rule. He appears in many other myths that we will address in this course. In the Iliad, Zeus is the arbitrator of events (he holds the scales which determine Hector must die) and he also presides over Odysseus' fate on his journey home in the Odyssey (see the council of the gods, Od. 1)


{\bf 4. Near Eastern Parallels}

{\bf Enuma elish}: see Powell chapter. Babylonian Succession myth featuring the god Marduk and his rise to power.

{\bf Kingship in Heaven}: Hittite succession myth that details story of Kumarbi, and the birth of storm-god Teshub from Kumarbi's body.

{\bf Kumarbi}: Servant of Anush who replaces him in succession for royal/divine power. Bites off Anush's genitals, swallows his semen, and becomes pregnant with the storm god Teshub, whom he gives birth to through the ``good place" (thought to be his penis).

{\bf Marduk}: The supreme god in Babylonian mythology, featured in the Nr. eastern text called the Enuma elish.

{\bf Song of Ullikummi}: Hittite myth which continues Kingship in Heaven. Tells of birth of stone child Ullikummi and his fight with the storm god Teshub.

{\bf Succession Myth}: A kind of myth which deals with the succession of sons over fathers over a number of generations. Many Greek and Near Eastern myths use this format to tell of how a divinity came to be a supreme ruler.


{\bf 5. The Myth of Prometheus and the Origin of Women}

{\bf Age of Heroes}: See `Ages of Man' (ID). Hesiod Works and Days. 
The group of mythological heroes who died in the Trojan and Theban wars.

{\bf Ages of Man}: Hesiod, Works and Days pp. 26-29. the five ages (also called `races') of mankind are Gold, Silver, Bronze, Heroic, Iron. You should know all of these individually for an ID, but when discussing one or another on an ID you should also be able to put them into context in relation to one another. Through the ages of man Hesiod tells a story of the gradual decline and generation of the human race and their lot upon the earth.

{\bf Bronze Age of Man}: See `Ages of Man' (ID). Hesiod Works and Days pp. 26-29. Hard, warlike group. Violent. Killed one another off in war. This group has nothing to do with the so-called ``Bronze Age" period of Greek history.

{\bf Elpis}: The Greek word for `hope' and `expectation.' Elpis is the only thing left inside Pandora's jar when she opens it in the Works and Days (lines 115 ff.)

{\bf Epimetheus}: Hesiod, Theogony, Works and Days. The slow-witted brother of Prometheus who accepts Pandora as a bride.

{\bf Golden Age of Man}: These men lived in the `golden age' - the time when men lived without toil or disease, under the rule of Kronos. 
Hesiod Works and Days (see also `Ages of Man' ID).

{\bf Iron Age of Man}: See `Ages of Man' (ID). The worst age for mankind. This is the age of toil, disease, good mixed with evil. Pandora seems to have in some ways set this age in motion by opening up her jar.

{\bf Pandora}: Hesiod, Theogony, Works and Days. Pandora is the first woman, and creates the race of women. She is manufactured by Zeus and the other gods in return for (and as punishment for) Prometheus' theft of fire. Pandora is married to Epimetheus and opens up the jar that she was given (and told not to open) with disastrous results for the history of human kind.

{\bf Prometheus}: Hesiod Theogony and Works and Days. Prometheus is a Titan who tricks Zeus twice, first through the fixing of the sacrifice, second by stealing fire back for men. In later traditions, such as Aeschylus' play Prometheus Bound, Prometheus is represented explicitly as a culture - hero (a little bit of extra info not mentioned in lecture). Although a god, Prometheus is often associated with man and humankind. He is punished by Zeus for his transgressions.

{\bf Silver Age of Man}: See also `Ages of man' (ID); Hesiod Works and Days.  Men lived as children for 100 years, and then died after a brief adolescence. Did not honor gods.


{\bf 6. Sexual Desire \& Finishing Pandora}

{\bf Aineias}: Son of Anchises and Aphrodite, Aeneas is a hero in the Trojan War.(You can spell his name either way).

{\bf Aphrodite}: Goddess of Love, born from the genitals of Ouranos in Hesiod's Theogony.She also features as the lover of Anchises and mother of Aineias in the Homeric Hymn to Aphrodite. See further her role in Homer's Iliad and - eventually - Virgil's Aeneid.

{\bf Danae}: Danae is the mother of the hero Perseus. Her father locks her up in a room, but Zeus still manages to impregnate her through a shower of gold.

{\bf Eos and Tithonos}: Eos (the goddess of Dawn) falls in love with the mortal Tithonos. She asks Zeus to grant him immortality but forgets to ask for ethernal youth...

{\bf Ganymede}: Beautiful boy whom Zeus falls in love with and takes up to Olympos to be his cup-bearer. Ganymede receives immortality and eternal youth.

{\bf Leda}: Zeus has sex with Leda in the form of a swan, and Helen is an offspring of their union.

{\bf Troy}: site of Judgment of Paris and the Trojan War. Located on west coast of Asia Minor (Turkey).


{\bf 7. Growing Up as a God}

{\bf Apollo}: Son of Zeus and Leto, sister of Artemis. Apollo's 3 timai are the bow, the lyre, and prophecy. Born in Delos, Apollo founds a sanctuary at Delphi.

{\bf Delphi}: Homeric hymn to Apollo. Site of Apollo's sanctuary and oracle in Northern Greece (see map in Athanassakis).

{\bf Hera}: Hera is the wife of Zeus and queen of the gods. Her special province is marriage. Ironically, her own marriage to Zeus is full of problems.

{\bf Hermes}: The trickster god who is born from Zeus and Maia. He begins life in relative obscurity but works his way into the Pantheon, where he takes on the honors or attributes (timai) of messenger, shepherd, and companion to mortals. He is also the god of boundaries and travel.

{\bf Homeric Hymns}: see Athanassakis (intro). A collection of hymns in honor of the gods, composed in the archaic period. They are called `Homeric' because they follow Homer's style, although were most likely not the work of the same author of the Iliad and/or Odyssey. The hymns celebrate the lives of the Olympian gods.


{\bf 8. Mothers and Daughters}

{\bf aetiological myth}: An example of a myth that explains something about a custom, ritual, or natural/scientific phenomena in ancient society, or even why something was named the way it was. 

{\bf Demeter}: Sister of Zeus and mother of Persephone (with Zeus). Founds mystery religion at Eleusis with Persephone. Goddess of agriculture and grain.

{\bf Demophoon}: Homeric Hymn to Demeter. Son of the mortal woman Metaneira, whom Demeter comes to in disguise as a mortal nursemaid. Demeter attempts to make him immortal and ageless by putting him in the fire every night and anointing him with ambrosia. Compare to other mortals who attain or almost attain divine status. 

{\bf Eleusynian Mysteries}: Homeric Hymn to Demeter. Secret mystery rites for initiates, located at Eleusis, and presided over by Demeter and Persephone.

{\bf Hades}: Brother of Zeus who is given the realm of the Underworld (also called Hades) by Zeus as his timE. Marries Persephone in the Homeric Hymn to Demeter.

{\bf Persephone}: Persephone is abducted into the Underworld by Hades in the Homeric Hymn to Demeter. She works out a compromise with her new husband so that she can spend 2 thirds of the year with her mother and a third underground with him. Persephone shares with Demeter the rites of the Eleusynian Mysteries.


{\bf 9. Iliad: Gods and Men}

{\bf Achilles}: Iliad. Greatest hero of the Achaeans, son of Peleus and Thetis. Argues with Agamemnon and sits out much of the fighting in the Iliad, because his honor has been insulted. Kills Hector in book 22.

{\bf Agamemnon}: Iliad; we will also encounter him in Odyssey books 1-4 and Aeschylus' Oresteia. Leader of the Achaeans (Greeks) at Troy, argues with Achilles. Brother of Menelaus.

{\bf Homer}: Iliad; Odyssey; Oral poet, composer of Iliad and Odyssey. Active in about the 8th century BCE (much about him is unknown). See further Muses, epic.

{\bf kleos}: Iliad, Odyssey. Greek word meaning fame, honor, reputation. a marker of heroic status.

{\bf Thetis}: Iliad; Theogony. Daughter of Nereus (old man of the sea), an immortal who is married to the mortal man Peleus and who gives birth to Achilles.


{\bf 10. Immortal but Troubled}

{\bf Hector}: Iliad. Greatest hero of the Trojans, the tragedy of his death closes the Iliad. Brother of Paris, killed by Achilles. (I will introduce Hector as an official ID on Wednesday as we have a few slides left to go from the Iliad). 

{\bf Hephaestus}: Iliad; Theogony; Works and Days; (later: Odyssey); Lame smith god of the Olympians, makes the Olympians laugh at the end of Iliad I. 

{\bf Sarpedon}: Iliad. Son of Zeus who dies in book 16 of the Iliad. Zeus is torn over whether to save his son or whether to let him die as fate has appointed.


{\bf 11. Back on Ithaca}

{\bf nostos}: Odyssey. `homecoming' (we get our word nostalgia, homesickness from this word). Word used to describe the returns of the Greek heroes from Troy after the city has been captured. The story of Odysseus' nostos to Ithaca is told in the first 12 books of the Odyssey, but in many ways his true nostos, his real return home, does not take place until the very end of the poem.

{\bf Orestes}: Son of Agamemnon and Clytemnestra who avenges his father's death as told in Book 1 of the Odyssey.

{\bf Phemius}: Oral poet (bard) in Odyssey who sings to keep the suitors entertained (book 1).

{\bf Telemachus}: Odyssey 1-4. Son of Odysseus and Penelope, just reaching age of manhood when Odyssey opens.

{\bf Telemachy}: Odyssey. A name given to the first four books of the Odyssey, because Odysseus does not appear until book 5. The Telemachy can be thought of as a `mini-Odyssey' - Telemachus' wanderings to different areas of Greece in search of news (more accurately the kleos) of his father.

{\bf Xenophanes}: (Dowden, ``Greeks on Myth" in course reader). Presocratic philosopher who criticized Homer and Hesiod for their portrayal of the gods (too immoral, too anthropomorphic).


{\bf 12. What's the Son of a Hero to Do? - Telemachus}

{\bf Eidothea}: Odyssey 4. Sea goddess. Daughter of Proteus (although she, like Telemachus in book 1, isn't entirely sure of this). Aids Menelaus in his journey home.

{\bf Helen}: Odyssey 4. Wife of Menelaus. 
Telemachus visits her in Sparta in the Telemachy. 
Known for her wiles as well as her beauty.

{\bf Menelaus}: Odyssey book 4. Brother of Agamemnon, husband of Helen. Telemachus visits him in book 4 of the Odyssey and Menelaus tells him about his nostos via Egypt.

{\bf Metrodoros}: ``Greeks on Myth". Allegorist from the 5th c. BCE who believed that the gods in Homer represented different parts of the human body and that the heroes represented different elements of the universe (sun, moon, etc.).

{\bf Nestor}: Odyssey book 3. Oldest of the Greek heroes who fought at Troy. Telemachus visits him in Pylos in Od. 3

{\bf Penelope}: Wife of Odysseus and mother of Telemachus. She is under increasing pressure to marry one of the suitors. See further Linda Pastan ``At the Loom," ``Rereading the Odyssey in Middle Age".

{\bf Proteus}: Odyssey 4; Old man of the Sea who can change into many shapes at will. Father of Eidothea.

{\bf Theagenes}: Presocratic physical allegorist from late 6th c. BCE. He believed that we should think of the gods in Homer as elements such as fire, water, air. See further Dowden, ``Greeks on Myth"


{\bf 13. Myth and the Fantastic}


{\bf Calypso}: Odyssey bk. 5. Goddess who inhabits island of Ogygia and who detains Odysseus there for a number of years.

{\bf Nausicaa}: Odyssey book 6. Phaeacian princess who is told in a dream by Athena to wash her clothes at the river in preparation for marriage. There she meets Odysseus.

{\bf Odysseus}: Greek hero of Homer's Odyssey. Marked by his cunning, cleverness, and ability to endure pain and suffering. Also appears in the Iliad. Favorite of Athena.

{\bf Phaeacians}: Odyssey books 6-13. Somewhat fantastic and otherworldy race of people who are nevertheless human. They entertain Odysseus and escort him back to Ithaca.

{\bf xenia}: Odyssey. Ancient Greek practice of guest-friendship.


{\bf 14. Telling Myths and Rationalizing Them}

{\bf allegory}: see Dowden and Graf. An ancient myth theory used by the Presocratics (eg. Theagenes, Metrodorus) and the Stoics, among others, to rationalize and interpret the hidden meaning behind myth.

{\bf Demodocus}: Od. 8. Blind bard among the Phaeacians on Scheria. Sings at the feast set up to honor the stranger Odysseus. See further Song of Ares and Aphrodite.

{\bf Schliemann}: Heinrich Schliemann was a late 19th century amateur archaoeologist who excavated at Troy and Mycenae, using Homer as his guide.

{\bf Song of Ares \& Aphrodite}: Od. 8. The adulterous affair between Ares and Aphrodite, and their capture by Aphrodite's husband Hephaestus, is told by Demodocus in book 8 of the Odyssey. Some ancient Greeks tried to rationalize this story though allegory.

{\bf Stoics}: Philosophers from the Hellenistic and Roman periods who continued the tradition of allegorizing Greek myth, especially through etymology (the meaning of words, or in this case, mostly the names of gods).


{\bf 15. Table Manners}

{\bf Aeolus}: Odyssey bk. 10. King who lives on floating island with his 6 daughter and 6 sons. Gives Odysseus a bag of winds to aid him in his passage home.

{\bf Laestrygonians}: Odyssey book 10. Fantastic giant people who spear Odysseus' men like fish when they pull into their harbor.

{\bf Lotus Eaters}: Odyssey 9. Group of people who feed Odysseus' crew the lotus, which causes them to forget their homeland.

{\bf Polyphemus}: Odyssey bk 9. Cyclops, son of Poseidon, who traps Odysseus in his cave and eats several of his men. Curses Odysseus when he finds out his name.

{\bf Poseidon}: Odyssey; God of the sea, brother of Zeus and Hades. Develops a hatred of Odysseus during his nostos.


{\bf 16. Meeting with the Dead}

{\bf Cattle of the Sun}: Odyssey 12. Immortal cattle of Helios whom Tiresias warns Odysseus not to eat, on any account. His men fall victim to their hunger and eat the cattle. The gods punish Odysseus' crew with a storm, killing all of them (except Odysseus) at sea.

{\bf Circe}: Odyssey book 10. Immortal daughter of Helios. Lives on an island visited by Odysseus and turns some of his men into pigs. She then sleeps with Odysseus and he lives with her for a year.

{\bf Elpenor}: Od 10-12. Member of Odysseus' crew who falls from the roof of Circe's house at the end of book 10, whom Odysseus meets in the underworld, and whom he buries on Aeaea at the beginning of book 12.

{\bf Scylla and Charybdis}: Odyssey 12. monsters whom Odysseus and his crew must pass by (and Odysseus must pass again after losing his crew). Scylla lives high up in a cliff on a rock face and eats several members of Odysseus' crew; Charybdis is a whirlpool who sucks everything down into herself and eventually spits it all back out.

{\bf Sirens}: Odyssey 12. Figures who sing alluringly to Odysseus and try to tempt him to pull in and listen to his song. Circe warns Odysseus that no-one who pulls in to the island ever leaves, which is why he fills his crews' ears with wax and has himself bound to the mast.


{\bf 17. Back at Argos}

{\bf Aegisthus}: Lover of Clytaemestra, plays a minimal role in killing of Agamemnon in Aeschylus. Aegisthus is the last remaining son of Thyestes (the only one not to be eaten).

{\bf Aeschylus}: Athenian playwright of the 5th century BCE (Classical period), wrote the Oresteia which consists of three plays: Agamemnon; Libation Bearers; Eumenides (aka ``The Kindly Ones").

{\bf Cassandra}: Aeschylus' Agamemnon. priestess of Apollo, daughter of Priam. Broke her word to Apollo and was cursed with knowing the truth but never being believed. Brought back as a war prize by Agamemnon and killed by Clyteamestra.

{\bf Clytemnestra}: In the Odyssey, left under the charge of a poet by husband Agamemnon when he went to war, but still seduced by Aegisthus (Od. 3. 255-300, p.36). In Aeschylus' Agamemnon, kills Agamemnon upon his return from war and is in turn killed by Orestes. Also plays an important role throughout the trilogy. Alternative spelling: Clytaemestra.

{\bf eagles and hare}: Aeschylus' Agamemnon. image used by the chorus in their first song to describe omen that made Artemis angry with pity. The eagles rip open the belly of a hare and eat its unborn young.

{\bf House of Atreus}: Aeschylus' Oresteia. household that begins with Tantalus but which is first officially cursed with Pelops and his dishonorable actions during the chariot race. Pelops' sons Atreus and Thyestes fight, continuing the curse, and it is then passed down, through Atreus, to his son Agamemnon (and his son after him... when will it ever end?).

{\bf Hubris}: A kind of pride associated with overreaching or overstepping one's boundaries. The effects of hubris are explored particularly in tragedy.

{\bf sacrifice of Iphigeneia}: Aeschylus' Agamemnon. Demanded by Artemis in response to seeing the omen of the eagles and hare, to atone for the spilling of innocent blood at Troy.


{\bf 18. What's the Son of a Hero to Do? - Orestes}

{\bf Classical Period}: See also archaic period; tragedy. Follows the Archaic period in Greece, often described as the `golden age' of Greek arts and literature. Athens is a very prominent city in the Classical period.

{\bf Clytaemestra's Dream}: Aeschylus, Libation Bearers. Clytaemestra dreams in the Libation Bearers that she has given birth to a serpent who bites her breast when she feeds it. The dream prompts her to send libations to the tomb of Agamemnon

{\bf Electra}: Aeschylus' Libation Bearers. Daughter of Agamemnon and Clytaemestra, helps plot with Orestes to kill Aegisthus and her mother.

{\bf Furies}: Aeschylus, Libation Bearers and Eumenides. These gruesome creatures are the avenging spirits who chase and torment Orestes after he kills his mother.

{\bf Libation Bearers}: The second play in the trilogy of the Oresteia, following the Agamemnon and coming before the Eumenides. This play takes its name from the chorus of women who bring libations to Agamemnon's tomb.

{\bf Tragedy}: An art form that developed in Athens in the Classical period, performed in a large open air theater, with a chorus and 2 or 3 actors (depending on the playwright) playing all of the characters. In the vast majority of cases, tragedy drew its subject matter from myth.


{\bf 19. Myth and the City (Aeschylus' Eumenides)}

{\bf Areopagus}: Aeschylus, Eumenides. The Areopagus was a rocky hill in Athens which gave its name to the court where homicide trials took place. This is also where Aeschylus has Orestes' crime tried in the last play of the trilogy. This is an Aeschylean invention to the myth.

{\bf Eumenides}: Aeschylus' Oresteia. In Greek this word means the ``Kindly Ones". It is the name given to the Furies at the end of the trilogy when Athena offers them a benevolent role in her city.

{\bf Furies}: Aeschylus, Libation Bearers and Eumenides. These gruesome creatures are the avenging spirits who chase and torment Orestes after he kills his mother.


\end{multicols*}
\end{document}
